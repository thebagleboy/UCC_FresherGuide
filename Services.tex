\zchapter{Services}
\begin{comment}
This Chapter provides an overview of UCC's services (as of February 2017); how to use them, what they are for, what servers are responsible for them. The full hostname for a server is \server{server.ucc.asn.au}.
Servers are usually named after fish beginning with M. This is because they are in the Machine Room, and they run Linux. The mascot for Linux is Tux, a penguin, and he likes to eat fish.
Remember that all services are maintained by UCC's members. If you are interested in learning more, or running a new service, ask someone!
\end{comment}
\newline
\newenvironment{uccservice}[1]
{
	\section{#1}
	%\begin{mdframed}
		%\noindent{\bf Machine(s) Involved:} \server{#2} % Freshers don't need to know this
	%\end{mdframed}
}
%================================================#
\newline
%================================================#
\begin{uccservice}{Drinks \& Snacks -- Dispense}
	UCC's most successful service is undoubtably the internet connected coke machine and not quite internet connected snack machine. These use serial communications to talk to merlo, which runs open source software written by talented members including John Hodge, Mark Tearle and David Adam. 
	A relay connected to merlo can be activated by door members from the snack machine to open the club's electronic door lock.
\end{uccservice}
%================================================#
\begin{uccservice}{Games}
	The Heathred A. Loveday memorial games server hosts many games including: Minecraft, TF2 and Wolfenstein: Enemy Territory (ET).
Administrator access to heathred is fairly unrestricted; it is also available as a general use server. For example, its GPU has been used in the past for number crunching projects.
\end{uccservice}
%================================================#
%\begin{uccservice}{Music}
%	TODO
%}
%\end{uccservice}
%================================================#
\begin{uccservice}{Email}
	UCC proudly runs its own mail server. You have an email account \texttt{<username@ucc.asn.au>}. Upon creating your account you can choose an address to foward all emails to. You can change this at any time by editing the ".forward" file in your home directory.\\
	A webclient is availiable at \url{http://webmail.ucc.asn.au} for ease of access. Other methods such as alpine for your own mail client work as well.
\end{uccservice}
	\begin{comment}
	% Normal people just use forwarding
	Alternately, you can use one of several methods to check your UCC email directly.
	\begin{enumerate}
		\item alpine --- Connect via SSH and run "alpine".
		\item webmail --- Several options will be presented to you at \url{http://webmail.ucc.asn.au}
		\item mail client (eg: Thunderbird) --- The server name is \server{secure.ucc.asn.au}. Use port 993 and IMAP. With your UCC username and password.
	\end{enumerate}
	\end{comment}
%================================================#
\begin{uccservice}{WebHosting}
	Members can publish their own sites! SSH to a server and edit the files in the directory "public-html". The website will appear at \url{http://username.ucc.asn.au}.
\end{uccservice}
%================================================#
\begin{uccservice}{FileStorage}
	With your account comes not one, but \emph{two} "home" directories for all your file storage needs. Both can be accessed through SSH, FTP, a UCC Windows or Linux machine, or from your phone if you have a FTP/SSH client installed. There is (currently) no enforced limit for how much you can store, but very large accounts are left out of our back-up server.
\end{uccservice}
%================================================#
\begin{uccservice}{VM Hosting}
	Members can get their own VM hosted at UCC by contacting someone on the wheel group to set it up for them.
	%\server{medico} runs the amazing ProxMox interface and is used for all new VMs. The typical way to use this interface is from a web browser on \server{maaxen}, a VM running on \server{medico}...
	%\server{heathred} is used for VMs when wheel complains that they aren't important enough to justify using all of \server{medico}'s CPU *cough* minecraft *cough*.
\end{uccservice}
%================================================#
%\begin{uccservice}{Develupment Environments}
%	TODO
%\end{uccservice}
%================================================#
%================================================#